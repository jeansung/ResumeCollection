% Resume Template 
% Created by Jean Sung
% Summer 2014
% find updates / submit feedback
% @ https://github.com/jeansung/ResumeCollection

% % Instructions & how to use
% 1) fill in resume header details 
% 2) create resume by using different section styles
% 3) copy and paste section as needed and fill in
%    relevant information
% Notes:
% -examples are at the bottom of the document, after all
%	of the section templates 
% -uncomment examples by pressing CMD/Control + U, 
%	recomment them by pressing CMD/Control + T
% -default font is 12pt, can be changed at top of LaTex
%	document, but remember that a different font means
%   different places for newlines if you don't want what's
%	flush with the left to interfere with date column
% -default setting for date column indentation explained 
%	below, can be changed

% % Start of Document % %
\documentclass[12pt]{article}
\usepackage{helvetica} % uses helvetica postscript font (download helvetica.sty)
\usepackage[margin= 0.5 in]{geometry}
\usepackage{hyperref}
\pagenumbering{gobble}
\normalsize

% % resume header details % %
\newcommand{\name}{\large\textbf{Gourav Khadge}} % Lovely sample name 
\newcommand{\addr}{some address}
\newcommand{\email}{\url{email addr}}
\newcommand{\phone}{phone number }

% % Information about Indent for Date Colum % %
% 8.5 inch wide paper (assuming standard US paper size)
% margin = 0.5 inch (can be changed at the top)
% date = 1 inch 
% 8.5 - margin - date Indent = indent for dates [6.5]
\newcommand{\aligndates}{\hspace*{6.5in}}

% % shortcuts & their descriptions % %
% header dot 
% used to seperate address, email and phone on
% the header of the document
\newcommand{\headerdot}{  $\bullet$  }

% vb
% used to seperate items in Section Style Gamma
\newcommand{\vb}{ $\mid$ }

% sectionNL the newline after a section title
\newcommand{\sectionNL}{\\[2pt]}

% custom tab & custom tabinline
% for use with section style beta
% the first line of a detail in beta is prefixed with \customtab
% any subsequent lines of a detail start with a \customtabinline
% after the newline command 
\newcommand{\customtab}{$\hspace{10pt}\bullet\hspace{2pt}$}
\newcommand{\customtabinline}{$\hspace{17pt}$}

\begin{document}

% Contact Information 
\begin{center}
\name \\
\addr \headerdot \email \headerdot \phone
\end{center}

% Section Style Alpha
% Suggested uses: Education, Awards, Honors, Activities
% Template: 
\begin{tabbing}
\aligndates\= \kill
{\large \textbf{Section Alpha Title} } \> \sectionNL
Subsection category 1 \> Month Year \\
Second Line \> \\ \\

Subsection category 2 \> Season \\
Second Line \> 
\end{tabbing}


% Section Style Beta
% Suggested uses: research / work/ volunteer experience
% Template: 
\begin{tabbing} 
\hspace*{6.5in}\= \kill
{\large \textbf{Section Beta Title} } \> \sectionNL
Activity Title 1, Location \>Dates \\
\customtab Detail 1 \\

Activity Title 2, Location \>Dates \\
\customtab Detail 2 example is really quite long because there are a lot of things I want to say \\ \customtabinline about this activity spills to the new line by using the customtablinline command
\end{tabbing}

% Section Style Gamma
% Suggested uses: relevant coursework 
% Template: 
\begin{flushleft}
{\large \textbf{Section Gamma Title }} \sectionNL
Item 1 \vb Item 2 \vb Item 3, etc 
\end{flushleft}

% Section Style Delta
% Suggested uses: Skills list 
% Template: 
\begin{flushleft}
{\large \textbf{Section Delta Title}} \sectionNL
\textit{Subsection 1:} item 1, item 2, item 3, item 4, etc
\end{flushleft}

% % Examples % %

%% Alpha Example 1:
%\begin{tabbing}
%\aligndates\= \kill
%{\large \textbf{Section Title} } \> \sectionNL
%Degree Title \> Month Year \\
%College or University, City, State \> \\ \\
%\end{tabbing}

%% Alpha Example 2
%\begin{tabbing}
%\hspace*{6.5in}\= \kill
%{\large \textbf{Awards and Honors} } \> \sectionNL
%Award 1 \> 1/12 -Present \\
%Award 2 \> Season Year \\
%\end{tabbing}

%% Alpha Example 3
%\begin{tabbing}
%\hspace*{6.5in}\= \kill
%{\large \textbf{Activities} } \> \sectionNL
%Activity 1 \> Dates done \\
%Activity 2 \> Season \\
%Activity 3 \> Years \\
%\end{tabbing}

%% Beta Example 1
%\begin{tabbing} 
%\hspace*{6.5in}\= \kill
%{\large \textbf{Research Experience} } \> \sectionNL
%Department of XYZ, Generic State School , City, State\>5/13 - 7/13 \\
%\customtab What you did  \\
%\customtab What technologies or lab techniques were used \\
%\customtab End accomplishment or final product
%\end{tabbing}
	
%% Beta Example 2
%\begin{tabbing} 
%\hspace*{6.5in}\= \kill
%{\large \textbf{Work Experience} } \> \sectionNL
%Job Title 1, Company, Claremont, CA \>9/13 -Present \\
%\customtab responsibilities and duties of the employee 
%\end{tabbing}

%% Beta Example 3 
%\begin{tabbing} 
%\hspace*{6.5in}\= \kill
%{\large \textbf{Volunteer Experience} } \> \sectionNL
%Volunteer Position 1, Organization \>9/12-2/13 \\
%\customtab duties and accomplishments as a volunteer 
%\end{tabbing}

%% Gamma Example 1 
%\begin{flushleft}
%{\large \textbf{Relevant Coursework}} \sectionNL
%Software Development \vb Domain Specific Languages \vb Computer Systems \\
%Discrete Mathematics \vb Data Structures/Program Development \vb Image Processing \\
%Programming Practicum \vb Principles of CS \vb Introduction to CS \vb Linear Algebra \\ 
%Differential Equations \vb Multivariable Calculus \vb Mathematics of Voting \\
%Introduction to Linguistics \vb Physical Chemistry 
%\end{flushleft}

%% Delta Example 1
%\begin{flushleft}
%{\large \textbf{Skills}} \sectionNL
%\textit{Programming:} C++, Java, Python, Racket, Prolog, MatLab, Unix/ Command Line, C, x86 Assembly\\
%\textit{Markup:} HTML/CSS, LaTeX \\
%\textit{Version Control:} Git, Subversion\\
%\textit{iOS Development:} Objective C, iOS, Cocoa Touch Framework, Cocos 2D, Box2D
%\end{flushleft}
\end{document}
